
Una vegada havia implementat la distància de Férchet per poder avaluar el rendiment dels models, vaig arribar a una clara conclusió:

Els models que tenen implementada la funció no lineal són més eficients. 

Afirmo això perquè els models sense la funció no arriben a estabilitzar-se, és a dir, no arriben al punt d'equilibri. En canvi, els models que tenen la funció si ho estan. Arribar punt d'equilibri ens garanteix que les imatges que generen seguin òptimes indefinidament. Per molt més que el model segueixi optimitzant-se, sempre donarà els mateixos resultats.

Els models que no presenten la funció, no assoleixen aquest punt a causa que la fidelitat de les imatges oscil·la, és a dir, el model pot arribar a generar imatges correctament, però no ho fa indefinidament, al cap d'unes interaccions la semblança de les imatges generades amb les reals ja no és significativa. Simplement, la qualitat de les imatges generades va oscil·lant entre bona i dolenta.

Aquest problema no és causat pel discriminador, això és perquè la variable independent de l'experiment és el circuit quàntic del generador, per tant, és el factor que té més possibilitat de ser el causant d'aquest comportament. No sé molt bé la causa exacta, però està clar que el mesurament parcial té un impacte. Tot i que hi ha excepcions, a vegades els models sense la funció no experimenten aquesta oscil·lació. Tanmateix cal remarcar que les vegades que passa són notablement més altes que les que no passa. Com es pot veure amb les dades de la taula \ref{tab:oscilations}.

\begin{table}[H]
	\resizebox{\textwidth}{!}{%
		\begin{tabular}{c|c|c}
			\hline
			& Presenta oscil·lació & No Presenta oscil·lació \\
			\hline
			Amb Funció No Lineal & 0 & 6 \\
			\hline
			Sense Funció No Lineal& 5 & 1 \\
			\hline
		\end{tabular}
	}
	\caption{Les dades provenen d'un total de 6 models, 3 d'ells amb un total de $700$ epoch i els altres 5 amb un total de $550$. El nombre d'interaccions no hauria d'afectar de cada manera les dades. Degut si hi ha una oscil·lació, es pot veure clarament a partir de les $400$ iteracions. Amb les dades es pot veure que és més probable que un model sense la funció lineal presenti una oscil·lació. Cal notar que cap model amb la funció ha tingut una oscil·lació. Les gràfiques que corresponen a cada model es poden veure en les figures \ref{fig:700_SD_score} i \ref{fig:550_SD_score}.}
	\label{tab:oscilations}
\end{table}

\begin{figure}[H]
	\begin{subfigure}[b]{.32\linewidth}
		\includegraphics[width=\linewidth]{figures/data/FD_score_1.png}
		\caption{}
	\end{subfigure}
	\begin{subfigure}[b]{.32\linewidth}
		\includegraphics[width=\linewidth]{figures/data/FD_score_2.png}
		\caption{}
	\end{subfigure}
	\begin{subfigure}[b]{.32\linewidth}
		\includegraphics[width=\linewidth]{figures/data/FD_score_3.png}
		\caption{}
	\end{subfigure}
	
	\begin{subfigure}[b]{.32\linewidth}
		\includegraphics[width=\linewidth]{figures/data/FD_score_A1.png}
		\caption{}
	\end{subfigure}
	\begin{subfigure}[b]{.32\linewidth}
		\includegraphics[width=\linewidth]{figures/data/FD_score_A2.png}
		\caption{}
	\end{subfigure}
	\begin{subfigure}[b]{.32\linewidth}
		\includegraphics[width=\linewidth]{figures/data/FD_score_A3.png}
		\caption{}
	\end{subfigure}
	\caption{Totes les gràfiques corresponen a models que s'ha executat al llarg de $550$ iteracions. Les figures \textbf{A}, \textbf{B} i \textbf{C}, corresponen a models sense la funció lineal. L'únic d'ells que no presenta una oscil·lació és el \textbf{C}. Les gràfiques sense la funció es poden comparar a les d'abaix, les quals representen models amb la funció implementada. Els models han estat creats per parelles, les quals estan organitzades verticalment. És a dir, les gràfiques \textbf{A} i \textbf{D} representen models que tenen els mateixos paràmetres inicials. El mateix passa amb \textbf{B} i \textbf{E} i amb \textbf{C} i \textbf{F}.}
\label{fig:550_SD_score}
\end{figure}

\begin{figure}[H]
	\begin{subfigure}[b]{.32\linewidth}
		\includegraphics[width=\linewidth]{figures/data/FD_score_4.png}
		\caption{}
	\end{subfigure}
	\begin{subfigure}[b]{.32\linewidth}
		\includegraphics[width=\linewidth]{figures/data/FD_score_5.png}
		\caption{}
	\end{subfigure}
	\begin{subfigure}[b]{.32\linewidth}
		\includegraphics[width=\linewidth]{figures/data/FD_score_6.png}
		\caption{}
	\end{subfigure}
	
	\begin{subfigure}[b]{.32\linewidth}
		\includegraphics[width=\linewidth]{figures/data/FD_score_A4.png}
		\caption{}
	\end{subfigure}
	\begin{subfigure}[b]{.32\linewidth}
		\includegraphics[width=\linewidth]{figures/data/FD_score_A5.png}
		\caption{}
	\end{subfigure}
	\begin{subfigure}[b]{.32\linewidth}
		\includegraphics[width=\linewidth]{figures/data/FD_score_A6.png}
		\caption{}
	\end{subfigure}
	\caption{Aquestes gràfiques corresponen a models que s'han executat al llarg de $700$ iteracions. Estan organitzades igual que les gràfiques de la figura \ref{fig:550_SD_score}. En aquests casos, com es pot observar tots els models sense la funció no lineal presenten les oscil·lacions. No obstant en la gràfica \textbf{A}, aquesta es molt feble. Per veure com afecten les oscil·lacions es pot veure la figura, on estan representades les últimes imatges que han generat els models que corresponen les gràfiques d'aquesta figura.}
	\label{fig:700_SD_score}
\end{figure}

En les gràfiques \ref{fig:550_SD_score} i \ref{fig:700_SD_score} es pot veure perfectament l'oscil·lació en la Distància de Fréchet. En aquest cas es pot interpretar aquesta mètrica com la semblança entre les imatges generades i les reals. Si aquesta mètrica convergeix a zero, es pot dir que el model a arribat al seu punt d'equilibri. Es pot veure com tots els models que tenen el mesurament parcial assoleixen aquest punt. D'aquesta manera demostrant l'eficàcia de la funció no lineal.

\begin{figure}[H]
	\begin{subfigure}[b]{.32\linewidth}
		\includegraphics[width=\linewidth]{figures/data/scatter_plot_4.png}
		\caption{}
	\end{subfigure}
	\begin{subfigure}[b]{.32\linewidth}
		\includegraphics[width=\linewidth]{figures/data/scatter_plot_5.png}
		\caption{}
	\end{subfigure}
	\begin{subfigure}[b]{.32\linewidth}
		\includegraphics[width=\linewidth]{figures/data/scatter_plot_6.png}
		\caption{}
	\end{subfigure}
	
	\begin{subfigure}[b]{.32\linewidth}
		\includegraphics[width=\linewidth]{figures/data/scatter_plot_A4.png}
		\caption{}
	\end{subfigure}
	\begin{subfigure}[b]{.32\linewidth}
		\includegraphics[width=\linewidth]{figures/data/scatter_plot_A5.png}
		\caption{}
	\end{subfigure}
	\begin{subfigure}[b]{.32\linewidth}
		\includegraphics[width=\linewidth]{figures/data/scatter_plot_A6.png}
		\caption{}
	\end{subfigure}
	\caption{Aquestes gràfiques corresponen als mateixos models que els de la figura \ref{fig:700_SD_score}. Les posicions de les gràfiques són les mateixes, per tant, si estan en la mateixa posició que les de l'altra figura, corresponen al mateix model. Les imatges generades pels models sense la funció no lineal (\textbf{A}, \textbf{C}, \textbf{B}) no s'assemblen a les reals. Es pot veure com els punts de les imatges generades (color violeta) no estan als mateixos valors. Mentre que en les gràfiques \textbf{D}, \textbf{E} i \textbf{F}, sí que ho estan. Els punts violetes sembla que representen la mitjana dels punts verds. També es pot observar que les imatges generades tendeixen a ser més variades que les reals.}
	\label{fig:700_images}
\end{figure}

Per poder veure la diferència entre les imatges directament, es pot veure la figura \ref{fig:700_images}, en la qual es mostren les 10 últimes imatges generades pels models optimitzats al llarg de 700 iteracions.

A partir de la Distància de Fréchet es pot apreciar l'efecte que té el mesurament parcial en la generació de les imatges, fent que aquest procés sigui més eficient i evitant una oscil·lació entre la generació d'imatges de bona i dolenta qualitat. Per tant, es pot afirmar que el mesurament té un efecte positiu en el model, corroborant la tesi exposada en Huang et. al. (2021) \cite{QGAN_exp}. 

No obstant això, es podria desenvolupar aquest experiment en major profunditat. No he tingut en compte el temps en el qual es tarden a generar les imatges. Els models amb el mesurament parcial implementat són més eficients, però en tenir més qubits és més costos simular-los i, per tant, tarden més temps per cada interacció. He fallat en mirar exactament quina és la diferència en termes de temps, tanmateix, ja que els models sense la funció no lineal no arriben a assolir el punt d'equilibri la gran majoria de les vegades, des d'un punt pràctic és recomanable utilitzar els models que si tenen la funció. 

A més a més es podria investigar la causa de l'oscil·lació en major profunditat. Deixant de banda la investigació teòrica de per què passa, es podria variar el nombre de qubits en els circuits, tant els ancilla, com el total de qubits del circuit, per veure quin és l'efecte que té tenir més qubits en el sistema ancilla. Potser investigaré pel meu compte aquesta qüestió. 

\section*{El camí que he recorregut}
Vull dedicar un petit espai per poder parlar que significa aquest treball per a mi, del tot l'esforç que he posat en aquest i finalment comentar tot el que he après al llarg d'aquest llarg camí.

Va dos anys que vaig començar a aprendre conceptes de computació quàntica, ha sigut un camí molt llarg, però que m'ha encant de recórrer. Comprendre tot l'ho que he après ha sigut una gran alegria, perquè són conceptes de camps que realment m'encanten (computació quàntica i intel·ligència artificial). A més a més, aprendre tot això m'ajudaria molt en cas que estudiï una carrera en física, que és el meu objectiu.

Deixant, d'una banda, els evidents beneficis d'avançar temari d'una carrera que vols fer en un futur, fent aquest treball he après altres coses extremadament útils.

Primer de tot, aquest treball està redactat en \LaTeX\footnote{\href{https://www.latex-project.org/}{La web oficial de \LaTeX}}, una espècie de llenguatge de programació per escriure equacions matemàtiques i documents de text. És àmpliament utilitzat per escriure articles científics en el món acadèmic, però més important és utilitzat en l'ambient educatiu de les ciències exactes per poder entregar deures i reports del laboratori. Tots els meus amics que estudien física ho fan servir.

A continuació tenim el fet de crear un gran projecte científic en Python, he après a programar fins al punt que crec que soc bastant bo, és en allò que soc millor. Programar en Python també és una habilitat que em serà útil a la universitat, perquè és el llenguatge de programació que es fa servir en la carrera de Física a la UB per realitzar experiments computacionals. Així mateix, vull mencionar que al llarg del procés de la creació del model van sorgir moltíssims problemes, en aquest document només he esmenat una part petita de la totalitat dels problemes que em vaig trobar al llarg del camí. Vaig estar més de mig any programant aquest model fins que estava tot perfecte per realitzar l'experiment. Definitivament, ha sigut la part més dura de tot el treball.

Per últim, he après a llegir articles científics sobre computació quàntics. Soc familiar amb la notació que s'empra en el camp i també el llenguatge específic que s'utilitza en els articles, amb això em refereixo a expressions en anglès. Tanmateix, no vull dir que puc interpretar els articles i valorar-los d'alguna manera, tan sols puc comprendre'ls, no obstant hi ha coses que no puc arribar a comprendre és clar, òbviament no tinc els mateixos coneixements d'un estudiant de postgrau.

Penso que aquest treball és el millor que he en la meva vida acadèmica, sigui en termes d'utilitat o en termes de realització personal, m'ha encantat fer-ho. He dedicat dos anys de la meva vida a això; des del primer moment que vaig veure que era la computació quàntica vaig tindre clar que faria el treball de recerca sobre aquest camp. I no me'n penedeixo de tot l'esforç que hi he posat.
