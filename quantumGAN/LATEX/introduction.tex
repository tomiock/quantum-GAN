Des de fa més de dos anys, m'he dedicat a estudiar computació quàntica durant el meu temps lliure. Buscava investigar un camp relacionat amb la mecànica quàntica, però sense que sigui molt complicat, que es pugui entendre a un nivell teòric i que m'entusiasmi.

La Computació Quàntica encaixa perfectament amb aquests criteris. És més senzilla que la mecànica quàntica pel fet que no està basada en càlcul o equacions diferencials, es basa en l'àlgebra lineal, utilitzant valors discrets, vectors i matrius. A més si es treballa a un nivell teòric senzill, no es tenen en consideració les interpretacions físiques, la qual cosa simplifica molt les coses. Com més m'endinsava, més ganes tenia de seguir.

La meva part favorita d'aquest camp és el \textit{Quantum Machine Learning} que consisteix a dissenyar i aplicar conceptes de \textit{Machine Learning} als ordinadors quàntics, com per exemple implementar quànticament les famoses xarxes neuronals, que estan darrere de la majoria d'intel·ligències artificials que veiem avui dia \cite{schuld:2014}.

QML és un camp de recerca jove i en creixement pel fet que els seus algorismes són ideals per a implementar-los amb els ordinadors quàntics actuals, els quals no són molt potents. Exemples d'aquestes implementacions serien [inserir aplicacions aquí], etc.

D'entre tots els tipus d'algorismes m'he centrat en les Xarxes Neuronals Quàntiques, anàlogues quàntiques de les Xarxes Neuronals tan utilitzades avui dia per a fer gran varietat de tasques. M'he interessat particularment en elles pel fet que tenia experiència en el passat amb les RNs clàssiques i havia vist que existeixen \textit{frameworks} de programari per a treballar amb elles com TensorFlow Quàntum \cite{tfq} que em podien ajudar.

Per a endinsar-me en el camp de QML, he hagut d'adquirir coneixements en àlgebra lineal, càlcul i física. Dins de QML en concret m'he dedicat a llegir papers que m'interessen i en un parell d'ocasions intentar implementar els algorismes detallats en aquests papers. Pot semblar una cosa impossible en principi pel fet que no tinc accés directe a un ordinador quàntic, no obstant això aquests no són necessaris pel fet que les operacions quàntiques poden ser simulades en un ordinador corrent d'escriptori (amb unes certes limitacions). Però puc tenir accés a ordinadors quàntics ja que IBM permet accedir als seus mitjançant \textit{IBM Quàntum Experience} \cite{IBM_Q}, encara que mai he donat ús d'això pel fet que no ho veia necessari.

En aquest treball de recerca m'he proposat implementar mitjançant codi un dels algorismes que he vist en un paper, una Xarxa Adversària Generativa Cuàntica (GAN, en anglès) \cite{GAN2014} que genera imatges a partir d'un circuit quàntic \cite{QGAN_exp}.

Com a pregunta a investigar m'he proposat verificar l'utilitat d'una funció no-lienal que implementen els autors en el circuit quàntic. Segons els autors de la xarxa aquesta funció millora el rendiment del model, es a dir, que les imatges són generades amb una major eficiència. 


